\documentclass{article}
\usepackage[utf8]{inputenc}
\usepackage[margin=1in]{geometry}

\begin{document}
\title{Implementing the Finite Difference Time-Domain Method for Solving Electromagnetic Problems}
\author{Thai Nguyen}
\date{\today}

\maketitle
\tableofcontents
\newpage

%------------------------------------------
\section{Introduction}
%------------------------------------------

%------------------------------------------
\section{Deriving the Update Equation}
%------------------------------------------
To simplify the derivations it is assumed that the materials are linear, isotropic, and non-dispersive. It is also assumed that no charges or currents are present.

With the assumptions stated, Maxwell's equations becomes
\begin{equation}
\nabla \cdot D(t) = 0
\end{equation}
\begin{equation}
\nabla \cdot B(t) = 0
\end{equation}
\begin{equation}
\nabla \times E(t) = -\frac{\partial B(t)}{\partial t}
\end{equation}
\begin{equation}
\nabla \times H(t) = J + \frac{\partial D(t)}{\partial t}
\end{equation}

and the constitutive relations
\begin{equation}
D(t) = \varepsilon E
\end{equation}
\begin{equation}
B(t) = \mu H
\end{equation}

where $H = H(\vec{r}, t)$, $E = E(\vec{r}, t)$

Using Cartesian coordinates the curl equations can be expanded
\begin{equation}
\left( \frac{\partial E_z}{\partial y}  - \frac{\partial E_y}{\partial z}\right)  \hat{a}_x - \left( \frac{\partial E_z}{\partial x}  - \frac{\partial E_x}{\partial z}\right)  \hat{a}_y + \left( \frac{\partial E_y}{\partial x}  - \frac{\partial E_x}{\partial y}\right)  \hat{a}_z = -\mu \left(\frac{\partial H_x}{\partial t} \hat{a}_x + \frac{\partial H_y}{\partial t} \hat{a}_y + \frac{\partial H_z}{\partial t} \hat{a}_z \right)
\end{equation}

\begin{equation}
\left( \frac{\partial H_z}{\partial y}  - \frac{\partial H_y}{\partial z}\right)  \hat{a}_x - \left( \frac{\partial H_z}{\partial x}  - \frac{\partial H_x}{\partial z}\right)  \hat{a}_y + \left( \frac{\partial H_y}{\partial x}  - \frac{\partial H_x}{\partial y}\right)  \hat{a}_z = \left(J_x + \varepsilon \frac{\partial E_x}{\partial t}\right) \hat{a}_x + \left(J_y + \varepsilon \frac{\partial E_y}{\partial t}\right) \hat{a}_y + \left(J_z + \varepsilon \frac{\partial E_z}{\partial t} \right) \hat{a}_z
\end{equation}

This can be separated into six equations

\begin{equation}
\frac{\partial E_z}{\partial y}  - \frac{\partial E_y}{\partial z} = -\mu \frac{\partial H_x}{\partial t}
\end{equation}
\begin{equation}
\frac{\partial E_x}{\partial z}  - \frac{\partial E_z}{\partial x} = -\mu \frac{\partial H_y}{\partial t}
\end{equation}
\begin{equation}
\frac{\partial E_y}{\partial x}  - \frac{\partial E_x}{\partial y} = -\mu \frac{\partial H_z}{\partial t}
\end{equation}

\begin{equation}
\frac{\partial H_z}{\partial y}  - \frac{\partial H_y}{\partial z} = J_x + \varepsilon \frac{\partial E_x}{\partial t}
\end{equation}
\begin{equation}
\frac{\partial H_x}{\partial z}  - \frac{\partial H_z}{\partial x} = J_y + \varepsilon \frac{\partial E_y}{\partial t}
\end{equation}
\begin{equation}
\frac{\partial H_y}{\partial x}  - \frac{\partial x_y}{\partial y} = J_z + \varepsilon \frac{\partial E_z}{\partial t}
\end{equation}

To further simplify the equations, it was assumed that there would be no moving charges present in the system $(J_x = J_y = J_z = 0)$. 

For the 1D case, it is assumed that the material extends infinitely in the x and y directions. This implies that the material looks the same when looked at the x-y planes. Expressed mathematically,
\begin{equation}
\frac{\partial}{\partial x} \rightarrow 0
\end{equation}
\begin{equation}
\frac{\partial}{\partial y} \rightarrow 0
\end{equation}

The six equations then become
\begin{equation}
\frac{\partial E_y}{\partial z} = \mu \frac{\partial H_x}{\partial t}
\end{equation}
\begin{equation}
\frac{\partial E_x}{\partial z} = -\mu \frac{\partial H_y}{\partial t}
\end{equation}
\begin{equation}
0 = \frac{\partial H_z}{\partial t}
\end{equation}

\begin{equation}
\frac{\partial H_y}{\partial z} = \varepsilon \frac{\partial E_x}{\partial t}
\end{equation}
\begin{equation}
\frac{\partial H_x}{\partial z} = \varepsilon \frac{\partial E_y}{\partial t}
\end{equation}
\begin{equation}
0 = \frac{\partial E_z}{\partial t}
\end{equation}

Assuming the source is a plane wave
\begin{equation}
E(z,t) = E f(k z - \omega t)
\end{equation}

The update equation then looks like the following
\begin{equation}
H_x = H_x + \frac{c_0 \Delta t}{\mu} \left( \frac{E_y - E_y}{\Delta z} \right)
\end{equation}
\begin{equation}
E_y = E_y + \frac{c_0 \Delta t}{\varepsilon} \left( \frac{H_x-H_x}{\Delta z} \right)
\end{equation}
%------------------------------------------
\section{Convergence and Validation}
%------------------------------------------
To check if the program works several test cases were devised
\begin{enumerate}
\item no pulse
\item pulse through space
\item pulse moving from free space to material with refractive index $n > 1$
\item pulse through coating
\item pulse through A-sandwich radome
\item pulse through periodic material
\end{enumerate}

For each case, the reflectance, R, and transmittance, T were recorded vs frequency. The energy conservation condition (R + T = 1) was also observed to ensure the program makes physical sense. The results of the simulated reflectance and transmittance were compared with the theoretical results to gain confidence in the program.

For a given pulse of the form
\begin{equation}
\vec{E} = \vec{A} \exp(\beta + \omega t)
\end{equation}

\subsection{No Pulse}


\subsection{Pulse through space}
For a system with a source travelling in free space the reflectance and transmittance are expected to be 
\begin{equation}
R = 0
\end{equation}
\begin{equation}
T = 1
\end{equation}
\subsection{Pulse moving from free space to material with refractive index $n>1$}
For a pulse travelling from free space to a material the reflectance and transmittance are expected to be
\begin{equation}
R = \left(\frac{n_0 - n_s}{n_0 + n_s}\right)^2
\end{equation}
\begin{equation}
T = 1 - R = \frac{4n_0 n_s}{(n_0 + n_s)^2}
\end{equation}

\subsection{Pulse through wall}
For a system with a wall of width $d$ and index of refraction $n_1$ separating two regions with indices of refraction $n_0$ and $n_2$ the reflection coefficient and transmission coefficient are
\begin{equation}
r = r_{01} + \frac{t_{01}t_{12}r_{12}}{\exp(j2\beta d) - r_{12}r_{10}}
\end{equation}
\begin{equation}
t = \frac{t_{01} t_{12}}{1-r_{12}r_{10}\exp(-j2\beta d)}
\end{equation}
where 
\begin{equation}
r_{01} = \frac{n_0 - n_1}{n_0 + n_1}
\end{equation}
\begin{equation}
r_{10} = \frac{n_1 - n_0}{n_1 + n_0}
\end{equation}
\begin{equation}
r_{12} = \frac{n_1 - n_2}{n_1 + n_2}
\end{equation}
\begin{equation}
t_{01} = \frac{2n_1}{n_0 + n_1}
\end{equation}
\begin{equation}
t_{12} = \frac{2n_2}{n_1 + n_2}
\end{equation}

%------------------------------------------
\section{Results and Comparison}
%------------------------------------------

%------------------------------------------
\section{Conclusion}
%------------------------------------------

%------------------------------------------
\section{References}
%------------------------------------------

%------------------------------------------
\section{Appendix}
%------------------------------------------

\end{document}

